% !TEX program = xelatex
\documentclass[12pt]{ctexart}
\usepackage[a4paper,margin=1in]{geometry}
\usepackage{iftex}
\ifPDFTeX
% pdfTeX: rely on ctex defaults for CJK
\else
\usepackage{fontspec}
% Use ctex Windows defaults; do not force specific fonts here
\fi
\usepackage{hyperref}
\hypersetup{colorlinks=true,linkcolor=blue,urlcolor=blue}
\usepackage{booktabs}
\usepackage{tabularx}
\usepackage{array}
\usepackage{graphicx}
\usepackage{enumitem}
\setlist{nosep}

\title{AI 群聊摘要助手:用户访谈计划与综合结果报告}
\author{团队:用户说得队\\成员:张天成、戴于皓、赵轩、李易涵、金杨洋}
\date{\today}

\begin{document}
\maketitle

\section{简介}
\subsection{感兴趣的领域与动机}
本项目聚焦于工作/兴趣群聊中的高效信息获取与协作。面对消息量爆炸、重要信息被冲散、决策与行动难以沉淀与追踪、跨时区成员难以跟进等痛点,我们提出“以主题卡、决策卡、行动卡、资源卡为核心的动态信息空间”,并支持点击溯源、编辑、订阅和一键执行。

\subsection{核心设想}
\begin{itemize}
  \item 以“\textbf{主题卡}、\textbf{决策卡}、\textbf{行动卡}、\textbf{资源卡}”结构化群聊要点;每个要点可回跳原消息(锚点高亮)。
  \item AI 主动推送 15–30s 可读的\textbf{微摘要卡};摘要作为\textbf{可操作对象},长度/风格/视角可调。
  \item 与日历/看板/飞书多维表等外部工具\textbf{双向联动},状态回写到群聊摘要。
\end{itemize}

\paragraph{通俗解释}
\begin{itemize}
  \item 把“没完没了的长聊天”整理成几张能直接操作的卡片,一眼看懂重点,还能点回原话核对上下文。
  \item 你可以只看和自己相关的内容,省时省力;需要时再展开细节,不被信息淹没。
  \item 做出的决定会自动带动任务流转,进展和结果能回写到群里,避免“说过就忘”。
\end{itemize}

\paragraph{四种卡片,怎么理解?}
\begin{itemize}
  \item \textbf{主题卡}:\emph{“这段时间大家在聊什么”}。自动把同一话题的消息聚在一起,并显示热度、参与人、时间窗;点击即可回到原消息片段。
  \item \textbf{决策卡}:\emph{“到底做不做,为什么,谁来做,什么时候做”}。明确结论、理由、约束、异议摘要、责任人/截止时间,并支持回滚提醒。
  \item \textbf{行动卡}:\emph{“可执行的待办”}。从对话里抓出任务、负责人、截止时间与依赖,一键推到看板/日历;完成/改期会回写到群聊摘要。
  \item \textbf{资源卡}:\emph{“重要链接/文件的合集”}。自动去重、保留最新版,打标签方便后续查找与复用。
\end{itemize}

\paragraph{举个小例子}
\begin{itemize}
  \item 讨论“登录页 A/B 测试”升温:系统生成\textbf{主题卡};当团队达成“采用版本B、周五上线”的结论,生成\textbf{决策卡};同步创建“更新埋点、回滚预案”等\textbf{行动卡};相关原型链接与埋点文档归入\textbf{资源卡}。
\end{itemize}

\paragraph{你会如何收到它(以及如何不被打扰)}
\begin{itemize}
  \item 订阅“只看我相关/关注主题/关键词/职责域”;系统在关键变更时推送 15–30 秒微摘要。
  \item 可设置静默时段与频率上限;也支持每日一封“日终 Digest”。
\end{itemize}

\paragraph{出错了怎么办}
\begin{itemize}
  \item 一键纠正主题归类/错误要点/术语误判,系统会学习本群的“黑话/缩写”,下次更准;所有改动都有审计记录。
\end{itemize}

\paragraph{隐私与可见性}
\begin{itemize}
  \item 只处理你允许的群与范围;卡片与字段可设可见性;所有重要操作可追溯(谁、何时、做了什么)。
\end{itemize}


\section{方法论:准备与执行}
\subsection{预设问题清单(通用 12+)}
\begin{enumerate}
  \item 你当前管理/参与的群聊类型与数量?各自目的是什么?
  \item 一天中你何时、如何处理群消息?为何这样做?
  \item 最近一次错过重要信息的场景?造成影响?
  \item 你如何确认“决策已生效”并通知到相关人?
  \item 你如何从群聊中提炼待办并跟踪进度?
  \item 群内常见术语/黑话有哪些?新人如何理解?
  \item 你用过哪些现有方案(置顶、公告、机器人、第三方看板)?为什么(不)好用?
  \item 什么情况下你会回溯消息?通常如何定位原始上下文?
  \item 你希望 AI 摘要以何种颗粒度/风格/视角呈现?
  \item 对隐私与可控性的担忧?期望哪些可见性/审计机制?
  \item 如果摘要有误,你会如何纠正与反馈?是否期待持续学习?
  \item 你希望能一键联动哪些外部工具(日历/多维表/看板)?
\end{enumerate}

\subsection{定制追问(按类型)}
\begin{itemize}
  \item 极端用户(社群运营):话题暴涨如何控噪?如何标注/合并主题?资源如何批处理与去重/版本化?
  \item 边缘化用户(听障):结构清晰与可键盘操作的需求;锚点回跳的准确性与语境窗口。
  \item 跨时区工程师:“我相关”的定义;离线微摘要与时段推送的偏好。
  \item 项目经理:决策卡字段是否满足审计;回滚提醒的触发条件。
  \item 领域专家:现有 IM/机器人能力边界;组织落地的采纳阻力与合规红线。
\end{itemize}

\subsection{分工(主持/记录)}
\begin{tabular}{@{}lll@{}}
\toprule
\textbf{访谈} & \textbf{主持} & \textbf{记录} \\
\midrule
直接用户1 & 张天成 & 戴于皓 \\
直接用户2 & 戴于皓 & 赵轩 \\
极端用户 & 赵轩 & 李易涵 \\
边缘化用户 & 李易涵 & 金杨洋 \\
领域专家 & 金杨洋 & 张天成 \\
\bottomrule
\end{tabular}

\subsection{采访执行(A/B/C)}
\paragraph{A. 时长}
每次 30–60 分钟,必要时分两段进行(观察+追问)。

\paragraph{B. 同意}
开场口头+书面同意,说明用途、保密、可随时终止;经许可方可录音/转写/截屏;数据匿名化与仅限课程作业使用。

\paragraph{C. 程序}
每次至少两名成员在场:\textbf{主持}(把控节奏、深挖动机)+\textbf{记录}(要点、锚点、截图),可使用录音/智能转写。分工详见“分工(主持/记录)”表;必要时全体参与。记住课堂练习中的开放式提问、复述与追问技巧。

\subsection{寻求参与者的建议}
\begin{itemize}
  \item 利用个人网络:以“朋友的朋友”为起点,注意避免过度使用直系亲友。
  \item 使用社交媒体和在线社区:发布招募帖,邀请二度人脉转介。
  \item 在合适场景中招募:在与研究领域相关的地点向合适对象尊重地提出邀请。
\end{itemize}

\section{采访结果记录}
\subsection{直接用户1(项目经理)}
\paragraph{访谈过程摘录}
\textbf{问}:最近一次你错过重要信息发生在什么场景?\\
\textbf{答}:上周评审里产品在群里确认了“版本B本周上线”,但我只看到局部消息,没意识到已定案,结果排期没同步到前端和数据。\\
\textbf{问}:你希望一个“决策卡”包含哪些内容?\\
\textbf{答}:结论、理由、影响范围、风险/回滚方案、责任人、截止时间,以及能跳回原始消息的链接。

\paragraph{关键点总结}
\begin{itemize}
  \item 决策需可审计与可追溯,强需求“决策卡+锚点回跳”。
  \item 待办提取目前手工,期望自动抽取并同步看板/日历。
  \item 需要上下游状态回写,减少沟通遗漏。
\end{itemize}

\subsection{直接用户2(跨时区工程师)}
\paragraph{访谈过程摘录}
\textbf{问}:你如何快速追上昨夜的讨论?\\
\textbf{答}:我固定早晚两个时间窗口看消息,但希望只看到“我相关”的主题变化和最终结论。\\
\textbf{问}:你对 AI 摘要的颗粒度有什么偏好?\\
\textbf{答}:先 15–30 秒微摘要,点击再展开要点与版本对比,必要时跳原消息。

\paragraph{关键点总结}
\begin{itemize}
  \item 强需求“只看我相关”与微摘要推送,适配时差。
  \item 主题聚类与版本对比帮助快速定位变化。
  \item 可离线阅读与稍后处理的能力有价值。
\end{itemize}

\subsection{极端用户(社群运营)}
\paragraph{访谈过程摘录}
\textbf{问}:管理 50+ 群时,什么最耗时?\\
\textbf{答}:同一话题在多个群升温时的去重与沉淀;还要判断哪个版本是最新且可信。\\
\textbf{问}:你希望资源如何管理?\\
\textbf{答}:自动去重与版本化,支持多群汇总;能批量 Pin、同步公告/活动页。

\paragraph{关键点总结}
\begin{itemize}
  \item 热度/参与人/时间窗是有效监控信号。
  \item 强需求资源卡去重/版本化与跨群汇总。
  \item 需要批处理能力提升运营效率。
\end{itemize}

\subsection{边缘化用户(听障分析师)}
\paragraph{访谈过程摘录}
\textbf{问}:什么会影响你理解群聊上下文?\\
\textbf{答}:没有结构的长段聊天;我需要清晰分块的要点和能回跳的锚点。\\
\textbf{问}:如果摘要出错你会怎么做?\\
\textbf{答}:希望能一键纠正并让系统学习群内术语,避免重复错误。

\paragraph{关键点总结}
\begin{itemize}
  \item 结构化摘要与可键盘操作对可达性关键。
  \item 锚点高亮回跳提升理解与信任。
  \item 纠错与持续学习闭环降低术语歧义。
\end{itemize}

\subsection{领域专家(企业 IM 运营)}
\paragraph{访谈过程摘录}
\textbf{问}:现有 IM 里的“机器人/公告”方案为何不足?\\
\textbf{答}:多是关键词触发或单向广播,难以承载主题级工作流与审计。\\
\textbf{问}:组织落地时的关键阻力?\\
\textbf{答}:权限/可见性/审计与合规,另一个是与现有工具的双向同步,避免信息孤岛。

\paragraph{关键点总结}
\begin{itemize}
  \item 需要可操作对象(主题/决策/行动)与完整审计轨迹。
  \item 与日历/看板等双向同步是采纳关键。
  \item 合规与可见性策略需内建到系统设计中。
\end{itemize}

\section{综合结果}
\subsection{初始用户需求列表(节选)}
以下条目均从访谈中总结而来,并与受访者观点相互印证:
\begin{itemize}
  \item 需要能够按“主题卡”聚合相关消息,并显示热度、参与人、时间窗指标(直接用户1、极端用户、专家)。
  \item 需要能够在“决策卡”中明确结论、理由、约束、异议摘要、责任人/截止时间(直接用户1、专家)。
  \item 需要一种方法从对话中稳定抽取“行动卡”(任务/负责人/截止/依赖),并可同步到看板/日历(直接用户1、跨时区)。
  \item 需要能够一键跳回摘要点对应的原消息片段(锚点高亮),保障可追溯性(边缘化、直接用户2)。
  \item 需要能够按“只看我相关/关键词/关注主题”订阅,并接收 15–30s 微摘要(跨时区、极端用户)。
  \item 需要一种方法实现资源卡自动去重与版本化,支持跨群汇总与批量处理(极端用户、专家)。
  \item 需要能够与日历/多维表/看板双向同步,状态回写到群聊摘要(多方共识)。
  \item 需要能够提供完善的权限/可见性/审计机制,满足组织合规(专家、项目经理)。
\end{itemize}

\subsection{聚焦的 4–5 条最有洞察力的需求}
\begin{enumerate}
  \item 需要能够形成可审计的“决策卡”,并与行动卡强绑定(项目经理/专家强诉求)。
  \item 需要能够实现“只看我相关”的订阅与 15–30s 微摘要推送(跨时区/高负荷用户)。
  \item 需要能够让摘要点一跳回原消息锚点,高亮上下文(可追溯性/信任基础)。
  \item 需要一种方法自动抽取行动卡并一键同步外部工具(闭环执行与回写)。
  \item 需要一种方法对资源卡去重/版本化并跨群汇总(极端用户面向规模化运维)。
\end{enumerate}

\section{结论与后续}
近期将基于上述 4–5 条需求制作可交互中保真原型(主题卡/决策卡/行动卡),进行 1–2 周 MVP 验证(增量主题更新、点击溯源、行动卡导出),并建立纠错与学习闭环(合并主题、纠正误分类、标记“是否决定”),以验证可用性与采纳路径。

\end{document}
