\textbf{A3}:\textbf{低保真模型}

团队:用户说得队

成员:张天成、戴于皓、赵轩、李易涵、金杨洋

2025年 11 月 13日

\textbf{原型一:可查证的决策卡}

纸质原型如图1:

\includegraphics[width=5.76111in,height=3.97361in,alt={微信图片\_20251113201523\_226\_219}]{./media/image1.jpeg}

图1

优点:关键信息分区清晰,AI 辅助减少输入成本,关联原文锚点满足追溯需求。

缺点:编辑页布局偏复杂,移动端操作可能拥挤;不同意见抓取精度依赖
AI,未考虑人工补充入口。

\textbf{迭代过程:}

迭代版本在原来的基础上优化的决策卡的编辑逻辑,新增了暂存和更新按钮,帮助用户更好对决策卡进行编辑和操作

\textbf{原型二:只看我相关微摘要}

纸质原型如图2:

\includegraphics[width=5.75694in,height=4.07986in,alt={微信图片\_20251113201416\_224\_219}]{./media/image2.jpeg}

图2

优点:订阅条件贴合用户需求,微摘要形式节省时间,操作路径短。

缺点:未提供摘要视角切换(个人 / 团队 /
家校)功能;推送频率未设置自定义选项,可能造成打扰。缺少个性化和自定义的功能。

\textbf{迭代过程}

迭代版本在原版的基础上增加了自定义功能,使得用户能在原来的基础上更加灵活的接收到自己想要的内容。

增加了切换视角的功能,用户能够从多个视角准确找到自己想要的内容。

优化了ui的逻辑,用户能够更加明了的了解内容。

\textbf{原型三:行动抽取与同步}

纸质原型如图3:

\includegraphics[width=5.76667in,height=4.18681in,alt={微信图片\_20251113201359\_223\_219}]{./media/image3.jpeg}

图3

优点:行动提取流程简单,支持多工具同步,状态回写满足闭环需求。

缺点:依赖长按触发提取,入口不够直观;未展示任务依赖关系,异常提醒机制未体现。

\textbf{迭代过程}

优化提取入口:在群聊输入框旁新增「+」号按钮,直接包含「提取行动卡」选项,入口更直观。

补充依赖关系展示:行动卡编辑页新增「添加依赖任务」选项,支持选择已有的行动卡建立关联,关联后在详情页以
``箭头'' 标注。

增加异常提醒:设置 ``截止前 1 天''\,``截止前 6 小时''
自动提醒,异常状态(如逾期未完成)以红色卡片形式推送至用户,支持点击「处理异常」快速反馈。
