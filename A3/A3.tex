% !TEX program = xelatex
\documentclass[12pt,notitlepage]{ctexart}
\usepackage[a4paper,margin=1in]{geometry}
\usepackage{iftex}
\ifPDFTeX
\else
\usepackage{fontspec}
\fi
\usepackage{xcolor}
\definecolor{teal}{RGB}{0,128,128}
\definecolor{gray}{RGB}{128,128,128}
\usepackage{hyperref}
\hypersetup{colorlinks=true,linkcolor=blue,urlcolor=blue}
\usepackage{booktabs}
\usepackage{tabularx}
\usepackage{array}
\usepackage{enumitem}
\setlist{nosep}
\usepackage{microtype}
\usepackage{graphicx}

\title{\textbf{A3:低保真模型}}
\author{
  团队:用户说得队 \\[0.5em]
  成员:张天成、戴于皓、赵轩、李易涵、金杨洋
}
\date{2025年 11 月 13日}

\begin{document}

\maketitle

\section{原型一:可查证的决策卡}

\subsection{设计思路 (Make)}

本原型旨在解决群聊中决策信息难以追溯和沉淀的问题。设计包含四个核心界面:

\begin{enumerate}
\item \textbf{决策卡展示页}:点击展开详情,支持快速预览决策内容
\item \textbf{决策卡编辑器}:分栏布局,左侧显示AI自动抓取的候选关键词,右侧显示对应原文锚点
\item \textbf{决策卡详情页}:展示决策的详细信息,包括结论、理由、责任人、截止时间等字段
\item \textbf{编辑页}:支持直接编辑各字段,每个字段都关联到对应原文,编辑后若与原文不符会标红提示
\end{enumerate}

生成机制:检测到"决定"等关键词时自动触发/用户手动触发,AI辅助填充关键信息。

\subsection{原型展示 (Show)}

\begin{figure}[htbp]
\centering
\includegraphics[width=\textwidth]{./media/prototype1.png}
\caption{原型一:可查证的决策卡(左侧为初版原型,右侧为迭代改进)}
\label{fig:prototype1}
\end{figure}

如图\ref{fig:prototype1}所示,原型包含:

\textbf{左侧 - 初版原型:}
\begin{itemize}
\item \textbf{决策卡预览界面}:简洁展示关键信息,支持点击展开详情
\item \textbf{决策卡编辑器}:双栏设计,左侧显示候选关键词(结论、理由、责任人等),右侧显示对应原文锚点
\item \textbf{决策卡详情页}:展示完整的决策信息,包含生成行动项、同步公告、修改记录等功能按钮
\end{itemize}

\textbf{右侧 - 迭代改进:}
\begin{itemize}
\item \textbf{编辑页优化}:展示详细的字段编辑界面,包括结论、理由、责任人、截止时间等
\item \textbf{字段关联}:每个字段都通过箭头关联到对应原文,可直接点击查看
\item \textbf{新增功能按钮}:底部增加"暂存"和"更新"两个按钮
\item \textbf{编辑提示}:可直接在字段中编辑,编辑后若与原文不符会标红或突出显示
\end{itemize}

\subsection{评估与反思 (Learn)}

\textbf{优点:}
\begin{itemize}
\item 关键信息分区清晰,便于快速定位
\item AI 辅助减少手工输入成本
\item 原文锚点关联满足追溯需求
\item 编辑后若与原文不符会标红,防止信息失真
\end{itemize}

\textbf{缺点:}
\begin{itemize}
\item 编辑页布局偏复杂,移动端操作可能拥挤
\item 不同意见抓取精度依赖 AI,未充分考虑人工补充入口
\item 初版缺少暂存功能,用户误操作风险高
\end{itemize}

\subsection{迭代改进}

\textbf{主要改进:}
\begin{enumerate}
\item \textbf{新增暂存和更新按钮}:允许用户暂时保存编辑进度,避免误操作导致信息丢失
\item \textbf{优化编辑逻辑}:提供更清晰的编辑流程指引
\item \textbf{修改记录功能}:追踪决策卡的历史变更,增强可追溯性
\end{enumerate}

\textbf{迭代理由:}通过初步展示和讨论,团队发现用户在编辑决策卡时容易因中断或误操作丢失内容。暂存功能可以让用户在未完全确定时先保存草稿,降低使用门槛。

\section{原型二:只看我相关微摘要}

\subsection{设计思路 (Make)}

本原型针对信息过载问题,提供个性化订阅和多层级摘要功能。核心设计包括:

\begin{enumerate}
\item \textbf{个人中心入口}:提供消息摘要、每日任务等功能模块,并设置红点标记的"设置"入口
\item \textbf{摘要草稿页}:包含搜索框和摘要卡片列表,支持勾选和筛选
\item \textbf{任务分类页}:支持按任务/时间划分,包含具体任务列表和进度更改功能
\item \textbf{设置功能页}:允许用户开启/关闭不同的订阅选项(个人、消息摘要、每日任务、每周总览、月度总览等)
\item \textbf{视角切换页}:支持切换个人/团队/公共频道等不同视角查看信息
\end{enumerate}

\subsection{原型展示 (Show)}

\begin{figure}[htbp]
\centering
\includegraphics[width=\textwidth]{./media/prototype2.png}
\caption{原型二:只看我相关微摘要(左侧为初版原型,右侧为迭代改进)}
\label{fig:prototype2}
\end{figure}

如图\ref{fig:prototype2}所示,原型展示:

\textbf{左侧 - 初版原型:}
\begin{itemize}
\item \textbf{图1 - 个人中心}:包含消息摘要、每日任务等功能入口,设置按钮用红色标记
\item \textbf{图2 - 摘要草稿页}:包含搜索框和摘要卡片列表,支持勾选和筛选功能
\item \textbf{图3 - 任务分类页}:支持按任务/时间划分,显示具体任务列表,支持进度更改和时间显示
\end{itemize}

\textbf{右侧 - 三个主要迭代改进:}

\textbf{修改1 - 订阅设置功能}(右上 左):
\begin{itemize}
\item 用户点击"设置"后进入配置页面
\item 可以开启/关闭不同类型的订阅:个人、消息摘要、每日任务、每周总览、月度总览、固定权限
\item 使用开关按钮控制功能启用状态
\item 让用户个性化选择订阅消息汇总方式
\end{itemize}

\textbf{修改2 - 视角切换功能}(右上 右):
\begin{itemize}
\item 点击个人头像后可切换视角
\item 支持"我"、"团队"、"公共频道"等不同视角
\item 每个视角显示对应的主要信息
\item 协作过程中查看不同角色的信息和任务
\item 注意:他人任务有权限访问限制,区分私密和公共信息
\end{itemize}

\textbf{修改3 - 标签与卡片分类}(右下):
\begin{itemize}
\item \textbf{标签功能}:左侧展示多个标签(label 1-4),每个标签可勾选并选择不同样式
\item \textbf{卡片分类}:右侧按"工作"、"学习"、"其它"等类别组织信息,每个分类显示对应的卡片
\item 支持自定义标签样式和分类管理
\end{itemize}

\subsection{评估与反思 (Learn)}

\textbf{优点:}
\begin{itemize}
\item 订阅条件贴合用户需求
\item 微摘要形式节省阅读时间
\item 操作路径短,一目了然
\item 支持按任务/时间等多维度查看
\end{itemize}

\textbf{缺点:}
\begin{itemize}
\item 初版未提供摘要视角切换(个人/团队/家校)功能
\item 推送频率未设置自定义选项,可能造成打扰
\item 缺少个性化和自定义功能
\item UI交互缺少更多个性化互动,如:
  \begin{enumerate}
    \item 标签设置
    \item 推送频率自定义
    \item 卡片分类
    \item 重要紧急程度分级
  \end{enumerate}
\end{itemize}

\subsection{迭代改进}

\textbf{主要改进:}
\begin{enumerate}
\item \textbf{增加自定义订阅功能}:用户可以在设置页面灵活选择需要的订阅类型(个人、消息摘要、每日任务、每周总览、月度总览等)
\item \textbf{新增视角切换功能}:用户能够从"我"、"团队"、"公共频道"等多个视角准确找到自己想要的内容,有利于团队协作和权限管理
\item \textbf{优化UI逻辑}:
  \begin{itemize}
    \item 增加标签系统,支持自定义标签样式
    \item 增加卡片分类(工作/学习/其它)
    \item 用户能够更加明了地了解和组织内容
  \end{itemize}
\end{enumerate}

\textbf{迭代理由:}初版设计虽然简洁,但缺乏灵活性。通过团队讨论发现,不同用户对信息摘要的需求差异很大:有的用户希望每日汇总,有的只关心@自己的消息,有的需要团队视角。因此迭代版本重点增强了个性化配置能力,让用户可以根据自己的工作习惯定制摘要方式。

\section{原型三:行动抽取与同步}

\subsection{设计思路 (Make)}

本原型解决从群聊到任务管理的自动化问题,实现行动卡的提取、编辑和多工具同步。核心设计分为三个部分:

\subsubsection{一、核心触发}
\begin{itemize}
\item \textbf{触发方式}:聊天界面长按消息,通过更多菜单选择[提取行动卡]
\item \textbf{智能识别}:AI自动识别"负责人/任务内容/截止时间/依据"并填入
\item \textbf{文本范围选择}:用户可以选中文本片段的开始和结束位置,确认或取消提取
\end{itemize}

\subsubsection{二、编辑行动卡}
简洁表单,支持多次识别结果。底部提供多种功能:
\begin{itemize}
\item \textbf{第一栏功能}:分享、收藏、编辑、删除、更多
\item \textbf{第二栏快捷操作}:昨天、今天、明天快速设置截止日期
\item \textbf{同步选项}:支持同步至日历、看板、表格等外部工具
\end{itemize}

\subsubsection{三、状态回写}
\begin{itemize}
\item \textbf{四种状态}:待处理、进行中、已完成、异常
\item \textbf{自动回写}:状态变更后自动生成简报信息,可选择同步到群聊
\item \textbf{搜索与筛选}:支持通过搜索栏快速定位特定行动卡
\end{itemize}

\subsection{原型展示 (Show)}

\begin{figure}[htbp]
\centering
\includegraphics[width=\textwidth]{./media/prototype3.png}
\caption{原型三:行动抽取与同步(左侧为初版原型,右侧为迭代改进)}
\label{fig:prototype3}
\end{figure}

如图\ref{fig:prototype3}所示,原型展示了完整的行动卡流程:

\textbf{左侧 - 初版原型的三个核心部分:}

\begin{itemize}
\item \textbf{一、核心触发}(左上):
  \begin{itemize}
    \item 聊天界面展示消息列表
    \item 长按消息后弹出提取窗口
    \item 用户可选择文本片段的开始和结束位置
    \item 提供"确认"和"取消"按钮
  \end{itemize}

\item \textbf{二、编辑行动卡}(左中):
  \begin{itemize}
    \item 简洁的表单界面,显示任务详细信息
    \item 底部第一栏:分享、收藏、编辑、删除、更多功能
    \item 底部第二栏:昨天、今天、明天快捷日期选择
    \item 支持同步至日历、看板、表格
  \end{itemize}

\item \textbf{三、状态回写}(左下):
  \begin{itemize}
    \item 搜索栏支持快速定位
    \item 四种状态:待处理、进行中、已完成、异常
    \item 状态变更后自动生成简报并同步
  \end{itemize}
\end{itemize}

\textbf{右侧 - 三个主要迭代改进:}

\begin{itemize}
\item \textbf{改进1 - 优化提取入口}(右上):
  \begin{itemize}
    \item 在群聊界面新增浮窗式「+」号按钮
    \item 浮窗直接显示行动卡列表
    \item 入口更加直观,无需长按即可访问
  \end{itemize}

\item \textbf{改进2 - 补充依赖关系}(右中):
  \begin{itemize}
    \item 行动卡编辑页新增「关联」按钮
    \item 支持选择已有行动卡建立依赖关系
    \item 可视化展示任务间的关联和依赖
  \end{itemize}

\item \textbf{改进3 - 状态回写细化}(虽未直接展示,但在说明中体现):
  \begin{itemize}
    \item 增加异常提醒机制
    \item 支持截止前自动提醒
    \item 异常状态以红色卡片突出显示
  \end{itemize}
\end{itemize}

\subsection{评估与反思 (Learn)}

\textbf{优点:}
\begin{itemize}
\item 行动提取流程简单,AI辅助降低手工输入成本
\item 支持多工具同步(日历、看板、表格),满足不同场景需求
\item 状态回写满足闭环需求,信息可追溯
\item 底部功能栏设计合理,常用操作触手可及
\end{itemize}

\textbf{缺点:}
\begin{itemize}
\item 依赖长按触发提取,入口不够直观
\item 未充分展示任务依赖关系
\item 异常提醒机制未体现
\item 初版缺少快捷入口,用户需要多步操作才能提取行动卡
\end{itemize}

\subsection{迭代改进}

\textbf{主要改进:}

\begin{enumerate}
\item \textbf{优化提取入口:}
  \begin{itemize}
    \item 在群聊输入框旁新增浮窗式「+」号按钮
    \item 直接包含「提取行动卡」选项
    \item 入口更直观,减少操作步骤
  \end{itemize}

\item \textbf{补充依赖关系展示:}
  \begin{itemize}
    \item 行动卡编辑页新增「关联」按钮
    \item 支持选择已有的行动卡建立依赖关系
    \item 关联后在详情页以箭头或流程图形式呈现
    \item 帮助用户理解任务间的前后依赖
  \end{itemize}

\item \textbf{增加异常提醒机制:}
  \begin{itemize}
    \item 设置「截止前 1 天」「截止前 6 小时」自动提醒
    \item 异常状态(如逾期未完成)以红色卡片形式推送至用户
    \item 支持点击「处理异常」快速反馈和调整
  \end{itemize}
\end{enumerate}

\textbf{迭代理由:}初版设计虽然功能完整,但在实际演示中发现:
\begin{itemize}
\item 长按操作对新用户不够友好,容易忽略该功能
\item 复杂项目中任务间有依赖关系,需要可视化展示避免遗漏
\item 缺少主动提醒机制,用户需要频繁检查任务状态
\end{itemize}

因此迭代版本重点优化了入口可见性、增强了任务关系管理,并引入智能提醒机制,使整个行动卡系统更加完善。

\end{document}
