% !TEX program = xelatex
\documentclass[12pt,notitlepage]{ctexart}
\usepackage[a4paper,margin=1in]{geometry}
\usepackage{iftex}
\ifPDFTeX
\else
\usepackage{fontspec}
\fi
\usepackage{xcolor}
\definecolor{teal}{RGB}{0,128,128}
\definecolor{gray}{RGB}{128,128,128}
\usepackage{hyperref}
\hypersetup{colorlinks=true,linkcolor=blue,urlcolor=blue}
\usepackage{booktabs}
\usepackage{tabularx}
\usepackage{array}
\usepackage{enumitem}
\setlist{nosep}
\usepackage{microtype}
\raggedbottom

\title{A2:准备与执行(需求提取、旅程分析、创意与投票)}
\author{团队:用户说得队\\成员:张天成、戴于皓、赵轩、李易涵、金杨洋}
\date{\today}

\begin{document}
\maketitle

\section{简介}
\subsection{团队与成员}
用户说得队:张天成、戴于皓、赵轩、李易涵、金杨洋。

\subsection{问题陈述}
当前群聊信息量大且结构松散,重要内容(主题、决策、行动、资源)被连续对话淹没,用户难以高效筛选与己相关的信息,导致决策难以沉淀、责任不清、任务闭环率低。
本项目通过引入结构化信息卡片(主题/决策/行动/资源)与智能摘要机制,提升信息可读性与可追溯性,并通过轻量确认流程增强任务执行闭环。

\subsection{领域与主题}
把“没完没了的群聊”整理成\textbf{四种卡片(主题/决策/行动/资源)}。支持\textbf{一键回到原文}、订阅“\textbf{只看我相关}”,还能和\textbf{日历/看板互相同步}。目标是:重要信息不丢、决定能落地、任务跟到底。


\subsection{用户画像}
\begin{itemize}
  \item 项目经理(办公协作):多人项目沟通,目标是“决定说清楚、能被查到、能带动执行”。
  \item 跨时区工程师(个人效率):固定时间窗口补看消息,目标是“只看我相关,快速知道有啥变化”。
  \item 社群运营(极端用户):管理 50+ 群,目标是“看热点、去重资源、批量处理事务”。
  \item 听障分析师(边缘化):依赖结构化摘要与回到原文的锚点,目标是“减少误解、提高可读性”。
  \item 企业 IM 运营专家(领域):关注权限/可见性/记录与联动,目标是“符合规则、能被组织接受”。
  \item 家长/班主任(家校群):作业/通知/会议管理,目标是“高触达+高确认,少打扰”。
  \item 社区居民/物业(社区群):报修/公告/缴费,目标是“闭环快、少重复问”。
  \item 兴趣社群组织者/成员:活动报名/签到/素材沉淀,目标是“报名转化高、资料不丢”。
  \item 志愿者/活动组织者:任务分发/场地协调/复盘,目标是“批量提醒、省时收尾”。
  \item 家庭/照护:购物清单/用药提醒/账单分摊,目标是“谁做什么一目了然”。
  \item 小微商家/团长:订单/发货/售后要点,目标是“状态回写、减少手动汇报”。
  \item 校园课程/社团:作业/课程变更/活动报名,目标是“提醒与统计清晰”。
\end{itemize}

\section{用户需求发现及用户画像}
\subsection{头脑风暴的用户需求}
\begin{enumerate}
  \item 需要能够按“主题卡”聚合相关消息,并显示热度/参与人/时间窗指标。
  \item 需要能够形成\textbf{可查证的“决策卡”},明确结论/理由/约束/不同意见/责任人/截止,并有\textbf{撤回/改动提醒}。
  \item 需要一种方法从群聊中自动抽取“行动卡”(任务/负责人/截止/依赖),并同步到看板/日历,状态能\textbf{发回群里}。
  \item 需要能够一键跳回摘要点对应的原消息片段(锚点高亮),保障可追溯性与信任。
  \item 需要能够订阅“只看我相关/关注主题/关键词/\textbf{身份}”并接收 15–30s 微摘要。
  \item 需要一种方法对资源卡\textbf{自动去重和保留最新版本},支持跨群汇总与批处理。
  \item 需要能够提供合适的权限/可见性/记录机制,满足组织/家庭/学校/社区的\textbf{规则/隐私要求}。
  \item 需要能够支持家校/社区/兴趣社群等生活场景的\textbf{报名/回执统计}与\textbf{公告发回群里}。
\end{enumerate}

\subsection{需求-画像-访谈映射}
\begin{tabularx}{\textwidth}{@{}p{3cm}p{3cm}p{3cm}X@{}}
\toprule
\textbf{需求} & \textbf{用户} & \textbf{场景} & \textbf{目标/依据} \\
\midrule
主题卡聚合+指标 & 项目经理/社群运营 & 版本评审、活动升温监控 & 收敛要点、看热度;访谈:直接用户1、极端用户 \\
可查证的决策卡 & 项目经理/IM 运营专家 & 评审定案、公告同步 & 结论清晰可追溯;访谈:直接用户1、专家 \\
行动卡抽取+联动 & 项目经理/跨时区工程师 & 排期分配、个人计划 & 少搬运、能落地;访谈:直接用户1、跨时区工程师 \\
锚点回跳溯源 & 听障分析师/跨时区工程师 & 核对上下文、历史复盘 & 快速回源、少误读;访谈:边缘化用户、直接用户2 \\
“我相关”订阅+微摘要 & 跨时区工程师/家长 & 通勤窗口、家校通知 & 只看相关、少打扰;A1 生活场景与访谈 \\
资源去重/版本化 & 社群运营/社区物业 & 跨群信息沉淀、公告回写 & 不重复、版本最新;访谈:极端用户、社区报修 \\
权限/可见性/记录 & IM 运营专家/班主任 & 可见范围、留痕导出 & 满足规则、风险可控;访谈:专家、家校 \\
报名/回执统计 & 家长/班主任/社团 & 家长会/活动报名 & 提高确认与统计效率;A1 家校/社团 \\
社区报修闭环 & 居民/物业 & 电梯维修、公共设施 & 闭环时长下降;A1 社区报修示例 \\
兴趣社群活动沉淀 & 组织者/成员 & 跑步/摄影活动 & 报名转化、成绩/素材沉淀;A1 兴趣社群 \\
志愿活动任务分发 & 组织者/志愿者 & 线下活动执行 & 批处理提醒、复盘便利;A1 志愿活动 \\
家庭清单与提醒 & 家庭成员/照护者 & 购物/用药/账单 & 行动卡联动日历、责任清晰;A1 家庭照护 \\
小微商家订单回写 & 团长/小微商家 & 拼团/二手交易 & 订单/售后要点与状态回写;A1 小微商家 \\
校园课程与社团 & 学生/老师/助教 & 作业/课程变更/报名 & 提醒准确、统计清晰;A1 校园课程/社团 \\
\bottomrule
\end{tabularx}

\subsection{精选的深层用户需求}
\begin{enumerate}
  \item \textbf{可查证的决策沉淀 + 撤回/改动提醒}:结论说清楚、能追溯、出问题能提醒处理,并能带动执行与公告。
  \item \textbf{“只看我相关”的订阅 + 15–30s 微摘要}:在固定时间快速看和我有关的变化与结论。
  \item \textbf{一键回到原文的锚点}:任何摘要都能回到原始上下文,减少误解。
  \item \textbf{行动抽取 + 外部工具同步}:从聊天到任务/日程自动化,状态能回写,尽量不手工搬运。
\end{enumerate}

\section{用户旅程分析与解决方案创意}
\subsection{需求1:可查证的决策沉淀 + 撤回/改动提醒}
\paragraph{用户旅程(阶段/行为/感受/需要/问题)}
\textbf{阶段:讨论阶段}\\
\textbf{行为}:参与群聊讨论,尝试收敛结论。\\
\textbf{情绪}:有点慌、混乱。\\
\textbf{需求}:收敛结论、不同意见被记录。\\
\textbf{痛点}:口头确认容易丢、结论不清。

\textbf{阶段:定案阶段}\\
\textbf{行为}:明确“谁/何时/做什么”。\\
\textbf{情绪}:小心、谨慎。\\
\textbf{需求}:一页看清所有关键信息。\\
\textbf{痛点}:公告/置顶不同步、责任不明。

\textbf{阶段:执行阶段}\\
\textbf{行为}:跟踪决策执行与潜在问题。\\
\textbf{情绪}:有压力、担心失控。\\
\textbf{需求}:触发条件到期提醒。\\
\textbf{痛点}:撤回信息分散、执行无闭环。
\paragraph{4+ 方案创意}
\begin{enumerate}
  \item AI 帮忙生成“决策卡”\textbf{草稿}(自动填结论/理由/约束/责任/截止),人最后确认 \textcolor{teal}{[AI 辅助]}。
  \item 决策->行动:定案就生成行动卡和公告草稿,并能\textbf{回写} \textcolor{blue}{[AI 自动化]}。
  \item 自动提醒撤回/复核:根据约束/风险监测触发条件,到点提醒 \textcolor{blue}{[AI 自动化]}。
  \item 不同意见集中看:把异议要点和锚点单列,避免被忽略 \textcolor{teal}{[AI 辅助]}。
\end{enumerate}

\subsection{需求2:“只看我相关”的订阅 + 15–30s 微摘要}
\paragraph{用户旅程}
\textbf{阶段:收件}\\
\textbf{行为}:窗口化查看群消息。\\
\textbf{情绪}:赶时间、焦虑。\\
\textbf{需求}:只看相关变更。\\
\textbf{痛点}:噪声多、信息泛滥。

\textbf{阶段:取舍}\\
\textbf{行为}:决定是否展开详情。\\
\textbf{情绪}:谨慎、不确定。\\
\textbf{需求}:版本对比/变更摘要。\\
\textbf{痛点}:上下文缺失、判断困难。

\textbf{阶段:处理}\\
\textbf{行为}:转成个人行动或忽略。\\
\textbf{情绪}:轻松、满足。\\
\textbf{需求}:一键收藏/稍后处理。\\
\textbf{痛点}:跨工具搬运繁琐。
\paragraph{4+ 方案创意}
\begin{enumerate}
  \item 智能订阅:按责任/提及/依赖/关键词/关注主题加权 \textcolor{blue}{[AI 自动化]}。
  \item 15–30s 微摘要卡:可切换视角(个人/团队/家校),一键展开 \textcolor{teal}{[AI 辅助]}。
  \item 时段策略 + 日终\textbf{汇总}:静默时间+每天一封总结(规则+AI 个性化) \textcolor{teal}{[AI 辅助]}。
  \item 稍后/离线阅读:自动打包关键上下文,断网也能看 \textcolor{teal}{[AI 辅助]}。
\end{enumerate}

\subsection{需求3:一键回到原文的锚点}
\paragraph{用户旅程}
\textbf{阶段:定位}\\
\textbf{行为}:从摘要回原文。\\
\textbf{情绪}:谨慎、专注。\\
\textbf{需求}:准锚点+小段上下文。\\
\textbf{痛点}:搜索耗时、定位不准。

\textbf{阶段:核对}\\
\textbf{行为}:确认措辞/假设/条件。\\
\textbf{情绪}:安心、放心。\\
\textbf{需求}:高亮与版本差异。\\
\textbf{痛点}:历史变化分散、易遗漏。

\textbf{阶段:复用}\\
\textbf{行为}:引用到公告/文档。\\
\textbf{情绪}:顺畅、高效。\\
\textbf{需求}:一键带回链。\\
\textbf{痛点}:复制粘贴易丢源。
\paragraph{4+ 方案创意}
\begin{enumerate}
  \item 自动定位原文片段:结合时间戳/相似度/对话回合找锚点 \textcolor{blue}{[AI 自动化]}。
  \item 上下文小窗:锚点前后若干条消息+参与者标签 \textcolor{teal}{[AI 辅助]}。
  \item 版本对比:主题/决策的演化轨迹可视化(规则+AI 辅助) \textcolor{teal}{[AI 辅助]}。
  \item 一键引用:把溯源链接带进公告/文档/看板(集成) \textcolor{gray}{[人工主导]}。
\end{enumerate}

\subsection{需求4:行动抽取 + 外部工具同步}
\paragraph{用户旅程}
\textbf{阶段:识别}\\
\textbf{行为}:从讨论里识别“谁/做什么/何时/依赖”。\\
\textbf{情绪}:专注、细致。\\
\textbf{需求}:自动提取关键元素。\\
\textbf{痛点}:表达不统一、易遗漏。

\textbf{阶段:编排}\\
\textbf{行为}:分配给人、设定截止与依赖。\\
\textbf{情绪}:有条理、控制感。\\
\textbf{需求}:标准化格式与同步。\\
\textbf{痛点}:在不同工具里重复操作。

\textbf{阶段:跟踪}\\
\textbf{行为}:进度更新与异常处理。\\
\textbf{情绪}:警惕、主动。\\
\textbf{需求}:实时状态与提醒。\\
\textbf{痛点}:状态不同步、异常未察觉。
\paragraph{4+ 方案创意}
\begin{enumerate}
  \item 任务抽取器:识别人名/动词/时间表达式,生成行动卡 \textcolor{blue}{[AI 自动化]}。
  \item 通用同步模块:日历/看板/表格标准化\textbf{双向同步}(集成) \textcolor{gray}{[人工主导]}。
  \item 自动回写:状态变化自动发回群聊摘要,异常提醒 \textcolor{blue}{[AI 自动化]}。
  \item 依赖可视化:显示关键路径/阻塞预警 \textcolor{teal}{[AI 辅助]}。
\end{enumerate}

\paragraph{AI 能力边界与失败模式}
AI 不负责“最终决策”,在证据不足、存在价值冲突或高风险场景时仅提供建议并强制人工确认;对召回不完整、幻觉与越权执行设置闸门(置信度阈值、灰度发布、可回滚、审计留痕)。

\section{解决方案投票与最佳方案}
\subsection{投票过程记录}
\paragraph{便利贴投票}
每名成员各 3 票,分别投给:\emph{可查证决策卡}、\emph{只看我相关+微摘要}、\emph{行动抽取+同步}。统计后这三项最高。

\paragraph{四类法再确认}
按“\textbf{最合理/最喜欢/最惊喜/最冒险}”四类把创意归类,做一次交叉检查:\emph{决策卡}(合理)、\emph{微摘要}(喜欢)、\emph{锚点回到原文}(安心)、\emph{撤回/复核提醒}(冒险远景)。

\subsection{最佳方案选择}
综合投票与四类法,确定本阶段\textbf{重点做}:\textbf{可查证的决策卡 + 只看我相关微摘要 + 行动抽取与同步} 三件套;\textbf{锚点回到原文}作为底层能力贯穿。

\end{document}
